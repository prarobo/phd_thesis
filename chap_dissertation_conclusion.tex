%=========================================================================================
% Conclusion

\chapter{Conclusions and Outlook}
\label{chap:thesis_conclusion}

%=========================================================================================

The work in this dissertation offers robotic image analysis tools designed to operate in noisy natural environments. 
Underwater environment is the primary domain used to validate these tools.
The two applications that drove the design of these object recognition tools are the automated subway car recognition and scallop recognition problems.
Additionally, a multi-view object classifier, that can operate in noisy images without segmentation, is also proposed.
Finally, an low cost \gls{rov}, CoopROV, was built to facilitate underwater experiments. The different insights gleaned from development of each of these tools will be discussed in the remainder of this section.

Chapter~\ref{chap:eigen} showcases the application of eigen-value based shape descriptors to the subway car recognition problem. 
Eigen-value based shape identification is a tool that can be used to identify simple geometric shapes. 
By reducing subway car recognition to rectangle matching, eigen-value based shape descriptor was
successful in recognizing subway cars from seabed images.
Though shape matching can aid object recognition, other cues like texture can often play an important role
in distinguishing an object. Since eigen-value shape descriptors only use shape information, 
they are not suited for applications where non-shape cues are more relevant in encoding object information. 
Additionally, while evaluating eigen-value shape descriptors, it was determined that 
discretization errors and segmentation errors can significantly impair the performance of eigen-value based shape descriptors.
Therefore, eigen-value shape descriptors is a useful tool for identifying
objects, provided we can guarantee that the shape of the object can be represented with minimal discretization error. 
Furthermore, for this method to work, the objects
we are interested in should exhibit a shape profile that is significantly different from
all other objects in the background. These requirements are very restrictive, and render recognition tasks involving 
complex shapes problematic. One such challenging problem is scallop recognition.
The visual cues presented by a scallop comprise its distinguishing crescent profile, along with its unique representative texture.
Since eigen-value descriptors neither handle complex shape profiles like crescents, nor do they capture textural information,
a more refined approach is warranted.
This led to the development of multi-layered object recognition framework proposed in Chapter~\ref{chap:scallop_recog}.

Marine biologists and oceanographers often depend on large natural image datasets to study benthic phenomenon. With the availability of robotic imaging vehicles, that generation datasets ranging millions of images are becoming increasingly common. Automated image processing tools are necessary to deal with such large image datasets. Most existing techniques are built on restrictive assumptions that do not generalize to noisy natural image datasets. The multi-layered object recognition approach described in Chapter~\ref{chap:scallop_recog} addresses this problem by providing researchers with a modular architecture that can scale to large natural image datasets. Modular architectures also contains the built-in flexibility to be reconfigured to solve varied object recognition problems.
This multi-layered framework was tasked with automated scallop recognition from \gls{auv} images, as a means to study scallop population. 
% This approach offers a hands-off automated alternative to manual enumeration that was also performed as a part of the scallop survey discussed in Chapter~\ref{chap:scallop_recog}.
% This four-layered automated scallop recognition framework combined a series of off-the-shelf computer vision tools like visual attention and graphcut, 
% along with custom designed noise filters and machine learning classifiers.
This framework was successful in recognizing 60\% of scallops in a dataset of over 8000 images.
The uniqueness of this approach lies in its ability to handle noisy natural images under varied environmental conditions.
To improve the performance of the classifier proposed here, and thereby reduce false positives, more information about a target object can be helpful.
To inject this additional information into the classification process, a multi-view object recognition algorithm discussed in Chapter~\ref{chap:distdes} was formulated.

The multi-view approach discussed in Chapter~\ref{chap:distdes} is machine learning 
technique designed for binary classification tasks. 
The objective here is to formulate a robust object classifier even in case of noisy image data.
Since noisy single-views of an object might often not contain sufficient information to unambiguously recognize it, information from multiple views are combined 
here to build a machine learning classifier. 
In this approach, the information from 13 images of each object specimen, captured from different heights, is encoded into a single object model.
Another salient character of this approach is the use of histogram-based global feature descriptors that do not
require segmentation of object pixels. Since segmentation can be challenging in noisy natural images, the lack of need for segmentation greatly boosts the appeal of this method. This method is evaluated on a combined dataset of 22 specimens comprising oranges and strawberries specimens.
All the specimens are correctly classified. Despite the small dataset of 22 images used, the exceptional performance of this classifier in noisy underwater data is encouraging. To decrease false positives, an approach like this could be used as a classification layer in a multi-layered approach, like the one described in Chapter~\ref{chap:scallop_recog}.

To collect data in underwater environments, a submersible robotic vehicle is often required.
Most commercially available robotic vehicles are either expensive or difficult to customize for research needs. To address this problem, CoopROV, a low cost \gls{rov} was designed as a research prototype to support underwater experiments. CoopROV carries an \gls{imu}, stereo cameras, depth and pressure sensors. The onboard electronics on the robot allows easy inclusion of new sensors. CoopROV also offers a \gls{ros} software interface that allows easy access to plethora of existing software tools. The low cost, and innate flexibility to modify the software and hardware, makes CoopROV an ideal platform to support underwater experiments.

There are some possible directions to extend the object recognition tools proposed in this dissertation. For instance the scallop recognition effort in Chapter~\ref{chap:scallop_recog} could benefit from using more descriptive 
templates. In Chapter~\ref{chap:scallop_recog}, we see that the appearance of a scallop is characterized by two crescents near the scallop rim: one bright and one dark. Currently, only the position of a dark crescent is captured by the templates used. Using templates that capture additional information, like the position of the bright crescent could be interesting. As for the multi-view object recognition method discussed in Chapter~\ref{chap:distdes}, a more expansive dataset could be used to test this algorithm. Another possible improvement is to generalize this multi-view algorithm to accept images from any series of heights instead of a fixed set of heights. Easing the restrictions on the different views needed by the multi-view algorithm could make the experiments and data collection effort easier. This multi-view approach could also be expanded to multi-class classification for dealing with more than 2 object classes. The CoopROV frame could be redesigned to improve the trim (weight distribution) of 
the vehicle. Designing controllers and building a localization system are other avenues where future improvements on CoopROV is possible.

This dissertation offers object recognition tools designed to work on noisy natural images. Each of these tools have been validated on different underwater applications. The developments here are intended to provide automated solutions to researchers dealing with large natural image datasets.
The availability of modular reconfigurable architectures to allow researchers to build custom solutions to solve their object recognition needs is the prime objective. The techniques proposed here are first steps in providing tools capable of handling noisy natural images. Further exploration on this domain is necessary to build more robust architectures that can recognize multiple classes of objects. Some examples of object recognition applications like subway car and scallop recognition are not all encompassing but provide an initial thrust in this domain. Further work to build extensive tools required by the marine research community is warranted. The overarching goal of this dissertation is to contribute to the object recognition 
tools available to strengthen the perception capabilities of robots. Ultimately, better scene understanding and perception abilities allow robotic systems to move towards autonomy.

