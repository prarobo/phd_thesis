%=========================================================================================
% Chap4: Literature review on underwater animal detection

\chapter{Underwater Animal Recognition}
\label{chap:animal_lit}

\section{Introduction}

In natural settings, living organisms often tend to blend into their environments to evade detection via camouflage. Webster's thesis work \cite{webster} provides a detailed exposition on the visual camouflage mechanisms adopted by animals to blend into their background. Under such circumstances of camouflage, there are very limited visual cues that can be used to identify animals. Even in the presence of visual cues, the task of identifying animals from natural scenes is shown to be a cognitively challenging and complex task \cite{wichmann}.

There is an extensive body of literature related to identifying marine animals. Some of this work is specialized for identifying different types of animals under specific environmental conditions. In general, the existing methods can be broadly divided into methods devised for identifying mobile organisms and methods for sedentary organisms. The former category is useful in dealing with a wide range of sea organisms like the varied species of fish that swim through water. The latter category is less studied. It includes identifying sedentary marine animals like scallops, corrals and sponges. Both categories present their own set of challenges. In the rest of this chapter we visit the techniques relevant to moving animals and show how they are different from the methods employed for sedentary animals. An overview of the existing literature on recognizing sedentary animals follows, with special emphasis on methods developed for identifying scallops. The shortcomings of the 
existing methods in recognizing sedentary underwater animals is addressed through a multi-layered method discussed in Chapter~\ref{chap:scallop_recog}.

%========================================================================================

\section{Methods for Recognition of Moving Underwater Organisms}

Recognizing and counting mobile marine life like fish~\cite{spampinato, edgington, williams} and studies in aquaculture \cite{zion} have been attempted. The recurring theme in these efforts involves the use of stationary cameras to detect the presence of moving species, provided that the background can be described by a prior model. This technique of assuming a known background, and using changes in the background as an evidence for the presence of a moving object entering the field of view of a sensor, is called background subtraction. In the marine species identification case, any changes to the background are assumed to be caused by a moving marine organism. The pixels in the image that deviate from the background model can be labeled as the pixels belonging to the organism.

Once a marine organism is detected through background subtraction, then other computer vision or machine learning techniques can be used to classify the organism into a specific species based on its visible characteristics. This classification task can be achieved through conventional machine learning approaches. For instance the salmon species classification algorithm developed by Williams et al. \cite{williams} uses active contours to model the shape of the fish before comparing these contours to known salmon species. However, if the pixels corresponding to the organism are contaminated by high levels of noise, a specialized technique that is robust to noise might be required.

Background subtraction requires a mathematical model that describes the distribution of background pixels. In an underwater setting, such a background model can only be obtained if the camera is stationary and is observing a static background, or in the case that the background model represents , the evolution of which over time can be captured through a mathematical model. Such well defined background models are not always available. An opportunity to employ the background subtraction-based techniques arises in underwater environments with stationary fixtures designed to study a specific underwater location. In instances where such stationary arrangement of cameras is not available, background subtraction is inapplicable due to the lack of a background model.

%========================================================================================

\section{Methods for Recognition of Sedentary Underwater Organisms}

Since sedentary marine animals like scallops do not typically move (unless chased by a predator), a mobile robotic platform is required to traverse subsea relief to image and recognize those marine animals. Extending background subtraction to work with mobile robotic platforms is challenging, since the motion of the platform causes changes in its background. Generating a model for the background to perform background subtraction in these cases is problematic. This makes the task of detecting sedentary organisms with moving sensors even more challenging than detecting moving organism with stationary sensors. The lack of background model in these cases motivates the development of a foreground model. If a foreground model is available, the task of detecting an organism can be realized as a search for pixels satisfying the foreground model in the image.

Detecting an organism typically involves segmenting all pixels of the organism, in order for one to classify the organism into a known category. 
The motion-based segmentation of marine animals that involves subtracting a known model of background from a snapshot of the environment, followed by attributing the pixels with non-zero values to the foreground is inapplicable in cases where the background model does not exist. 
Furthermore, the task of segmentation can be challenging in noisy images with weak edges, since the boundary pixels of the foreground object cannot be easily distinguished from background pixels.

Thus, the lack of background model makes background subtraction problematic. This leads to the need for techniques that depend on foreground models, and use of other features to detect and segment organisms from the background. This task becomes even more complicated if the organism does not present significant visual cues that make it distinctive from the background, as in the case of creatures exhibiting camouflage. High levels of noise or unpredictable environmental variables could also significantly affect the effectiveness of any animal recognition mechanism.

%========================================================================================

\section{Scallop Recognition}

Previous efforts to detect animals like plankton~\cite{mcgavin_plankton, stelzer_rotifier}, clam~\cite{forrest_clam} and a range of other benthic megafauna~\cite{schoening} exist. Most of these methods here are specialized to a specific species, or only tested in controlled environments. In some cases, the methods require specialized apparatus (like in the plankton recognition studies~\cite{mcgavin_plankton, stelzer_rotifier}). 
A series of automated tools like specialized color correction, segmentation and classification modules along with some level of manual expert support, can be combined identification of several marine organisms like sea anemones and sponges from natural image datasets~\cite{schoening}. 

There are several aspects that make scallop recognition challenging.
Scallops, especially when viewed in low resolution, do not provide features
that would clearly distinguish them from their natural environment.  This
presents a major challenge in designing an automated identification process 
based on visual data.  To compound this problem, visual data collected
from the species' natural habitat contain a
significant amount of speckle noise.
Some scallops are also partially or almost completely
covered by sediment, obscuring the scallop shell features.
A highly robust detection mechanism is required to overcome these impediments.

There is a range of previously developed methods specialized for scallop recognition 
\cite{dawkings13,guomundsson,enomoto9,enomoto10,fearn, prasanna_med, prasanna_aslo, prasanna_igi} 
that operate on different assumptions, either with regards to the environmental conditions or the quality of data.
Existing approaches to automated scallop counting in artificial environments
 \cite{enomoto9, enomoto10} employ a detection mechanism based on intricate distinguishing features 
like fluted patterns in scallop shells and exposed shell rim of scallops, respectively.
Imaging these intricate scallop shell features might
be possible in artificial scallop beds with stationary cameras and 
minimal sensor noise, but this level of detail 
is difficult to obtain from low resolution images of scallops in their natural environment. 
A major factor that contributes to the poor image resolution is the fact that sometimes the image of a target
is captured several meters away from it. 
Overcoming this problem by operating an underwater vehicle much closer to the ocean floor 
will adversely impact the image footprint (i.e. area covered by an image) and increase the risk of damaging the vehicle.

Furthermore,  existing work on scallop detection \cite{dawkings13, guomundsson} in their natural
environment is limited to small datasets (often less than 100 images). 
A sliding window approach has been used \cite{guomundsson} to focus the search for the presence of scallops. The large number of overlapping windows that need to be processed per image raises scalability concerns if this method were to operate 
on a large dataset containing millions of images. Additionally, the small number of natural images used as a test set raises questions about the generalizibility of this method and its ability to function under varied environmental conditions.
The work by Dawkins \cite{dawkings13} is more detailed in its treatment of the natural environmental conditions spanning the scallop habitat. The images used here are collected using a towed camera system that minimizes noise, a fact which greatly enhances the performance of the machine learning and computer vision algorithms. Despite the elaborate imaging setup designed to minimize noise, the results reported are derived only from a few tens to hundreds of images.
It is not clear if those machine learning methods \cite{dawkings13} can extend to noisy image data captured by \gls{auv}s.
From these studies alone, it is not clear if such methods can be used effectively
in cases of large datasets comprising several thousand seabed images.
An interesting example of machine-learning methods applied to the
problem of scallop detection \cite{fearn} 
utilizes the concept of \gls{buva}.
The approach is promising but it does not use any ground truth for validation.  

There is more work \cite{prasanna_med, prasanna_aslo, prasanna_igi} that offers a multi-layered object recognition framework validated on
a natural image dataset for scallop recognition application. 
The main emphasis there (and in this dissertation) is to develop a technique that can work on low quality noisy sensor data collected using \gls{auv}s.
The other objective is to build a scalable architecture that can operate on 
large image datasets in the order of thousands to millions of images
and can be generalized for recognizing other marine organisms.
A detailed account of this multi-layered approach is discussed in Chapter~\ref{chap:scallop_recog}.

%========================================================================================

\section{Applicability of Existing Methods for Scallop Recognition}

Existing methods for recognizing moving underwater organisms like fish depend on background subtraction. Though background subtraction 
offers a convenient way to detect the presence of an animal along with motion based segmentation to isolate the pixels of the organism, the need for a well defined background model limits its applicability. A well-defined background model is only obtainable in scenarios where the background is unchanging, or when the evolution of the background can be captured by a mathematical model. Such modeling is possible if the imaging setup is stationary and the \gls{fov} of the predominantly captures a static background. In such a case any moving object can be identified and isolated from the rest of the background. However, this is only possible when the imaging system is expressly looking for moving objects. In case of sedentary organisms, the imaging system has to move to capture data and identify the organisms from the images. Then, the background distribution becomes very challenging, if not impossible, to model.

In order to recognize sedentary organisms, a mobile imaging system capturing images is required. Stationary camera systems have been used to identify scallops from artificial beds \cite{enomoto9, enomoto10}. However in these cases, the region to be imaged was small enough to fit inside the \gls{fov} of a network of stationary cameras. The environment was an artificial scallop bed of known substrate, and parameters like illumination were fully controlled. In natural environments, the magnitude of the region to be imaged is sufficiently large to make it prohibitively expensive to use a large array of stationary cameras. Mobile sensors offer a cost effective and practical alternative to this problem. The use of a mobile camera system however exposes the system to a gamut of environmental conditions. The bifurcation of the gathered data into foreground and background without full knowledge of the changing environmental factors poses a technical challenge. The methods developed for identifying 
organisms under artificial controlled conditions are not transferable to instances of systems that operate over natural image datasets.

A few automated scallop recognition systems that have been validated on natural image datasets \cite{dawkings13,guomundsson,fearn, prasanna_med, prasanna_aslo, prasanna_igi}. Still, those that are validated on natural image datasets are often limited to small test sets of less than a few hundred images \cite{dawkings13,guomundsson,fearn}. The significant cost and effort required to collect natural image data explains the reason for the limited amount of work in this domain. Furthermore, developing an automated system to recognize organisms from natural images also entails a manual annotation effort to obtain a generalizable representation of a species that spans different variations of its habitat. Such a labeled dataset is intended to serve as a knowledge-base for a machine learning system developed to recognize an organism. 

The scallop recognition approach in \cite{prasanna_med, prasanna_aslo, prasanna_igi} is designed to deal with noisy \gls{auv} images that are characterized by high levels of speckle noise, uneven illumination and low contrast. This multi-layered architecture that has been validated over a dataset containing a few thousand images. A more detailed comparison of Dawkins et al. \cite{dawkings13} against Kannappan et al.\cite{prasanna_igi} along with working details of the latter is discussed in Chapter~\ref{chap:scallop_recog}.

%========================================================================================

\section{Motivation for a Generalized Automated Object Recognition Tool}

Understanding the parameters that affect the habitat of underwater organisms is of interest to marine
biologists and government officials charged with regulating a multi-million dollar fishing industry. Dedicated
marine surveys are needed to obtain population assessments. One traditional scallop survey method, still
in use today,  is a dredge-based survey. Dredge-based surveys have been extensively used for scallop population density
assessment \cite{nefsc}. The process involves dredging part of the ocean floor, and manually counting the
animals of interest found in the collected material. In addition to being invasive and
detrimental to the creatures’ habitat \cite{jenkins}, these methods have accuracy
limitations and can only generalize population numbers up to a certain extent.
There is a need for non-invasive and accurate survey alternatives.

The availability of a range of robotic systems in form of towed camera and Autonomous Underwater Vehicle
(auv) systems offer possibilities for such non-invasive alternatives. Optical imaging surveys using underwater
robotic platforms provide higher data densities. The large volume of image data (in the order of thousands
to millions of images) can be both a blessing and a curse. On one hand, it provides a detailed picture of
the species habitat; on the other requires extensive manpower and time to process the data.
While improvements in robotic platform and image acquisition systems have enhanced our capabilities to
observe and monitor the habitat of a species, we still lack the required arsenal of data processing tools. This
need motivates the development of automated tools to analyze benthic imagery data containing scallops.

One of the earliest video based surveys of scallops \cite{rosenkratz} reports that
it took from 4 to 10 hours of tedious manual analysis in order to review and process one hour
of collected seabed imagery. The report suggests that an automated computer technique for
processing of the benthic images would be a great leap forward; to this time, however, no
such system is available. There is anecdotal evidence of in-house development efforts by the
HabCam group \cite{gallager} towards an automated system but as yet no such
system has emerged to the community of researchers and managers. A recent manual count
of our AUV-based imagery dataset indicated that it took an hour to process 2080 images,
whereas expanding the analysis to include all benthic macro-organisms reduced the rate
down to 600 images/hr \cite{walker}. Another manual counting effort \cite{oremland} 
reports a processing time of 1 to 10 hours per person to process each image tow
transect (the exact image number per tow was not reported). The same report indicates that
the processing time was reduced to 1–2 hours per tow by subsampling 1\% of the images.

Future benthic studies can be geared towards increasing data densities with the help of robotic optical surveys.
It is clear that the large datasets, in the order of millions of images, generated by these surveys will impose 
a strain on researchers if the images are to be process manually. This strongly suggests the need for automated tools
that can process underwater image datasets. Motivated by the need to reduce human effort, Schoening \cite{schoening} has proposed a range of tools that
can be generalized to organisms like sea-anemones. With an additional requirement of being able to work with low-resolution noisy underwater images, 
a generalized multi-layered framework that can be used to detect and count underwater organisms has been proposed \cite{prasanna_med, prasanna_aslo, prasanna_igi}. This method has been evaluated on a scallop population assessment effort on a dataset containing over 8000 images, the details of which can be found in Chapter~\ref{chap:scallop_recog}.

%========================================================================================