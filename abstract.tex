Building autonomy into robotic systems enables them to operate 
unsupervised over remote or potentially dangerous domains.
Object recognition is an important trait required for a robotic system to achieve autonomy.
The task of object recognition is essentially understanding and labeling the different components in a robot's environment. This task can get complicated for robots that operate in unstructured natural environments, like forests or deep sea, due to noise in sensor measurements. Noisy sensor measurements can potentially affect a robot's perception of the world. To avoid being misled by noise in sensor measurements, robots need to possess robust object recognition capabilites that can handle noise in sensor measurements. Such robust object recognition capabilities are valuable for processing large natural image datasets. Underwater image datasets gathered by marine scientists and oceanographers, is one such case where object recognition capabilities could be invaluable. The objective here is to analyse these datasets to understand natural phenomenon, for instance recognize organisms. Sifting through such \emph{big} datasets, ranging millions of images, to make inferences is growing to be big challenge to the 
research community. This motivates the need for automated object recongition and image analysis tools.
This dissertation focusses on object recognition techniques capable of operating in noisy natural environments.
A series underwater object recognition  problems have been solved as means to validate the developed object recognition algorithms. 
Each technique was developed to complement the shortcomings of the existing tools available to the research community. 

To start with, eigen-value based shape descriptor were tasked to solve the subway car recognition problem. Despite the success in solving this problem, eigen -value shape descriptor cannot leverage textural cues for object identification. This primary drawback, among other shortcomings, lead to the development of a multi-layered object recognition architecture. This multi-layered architecture was tested on an scallop enumeration problem. Over 60\% of scallop instances were successfully identified. To improve the machine learning classifier of this multi-layered framework, and also to minimize false positives, a multi-view object classication approach is proposed. This multi-view approach combines histogram-based global cues from a series of images of a target, captured from different heights, to construct a machine learning classifier. This multi-view method was successful in classifying all specimens in the available dataset. In addition to the developed object recognition methods, a low cost \gls{rov}, 
named CoopROV, was designed for underwater data collection to support research experiments.
