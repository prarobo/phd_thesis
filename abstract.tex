Autonomous robotic systems can operate in an unsupervised manner over remote or potentially dangerous domains.
Object recognition is an important trait required for a robotic system to achieve autonomy.
The task of object recognition involves understanding and labeling the different components in a robot's environment. This task becomes complicated for robots that operate in unstructured natural environments, like forests or deep sea, due to noise in sensor measurements. Noisy sensor measurements can potentially affect a robot's perception of the world. To avoid being misled by corrupted measurements, robots need to possess robust object recognition capabilities that can handle noise in sensor measurements. Such robust object recognition capabilities are valuable for processing large natural image datasets. One such case of image datasets are the underwater imagery data gathered by marine scientists and oceanographers; there, automatic object recognition capabilities could be invaluable. Such a capability would enable the automatic analysis of these datasets to understand natural phenomena, for instance to recognize different organisms of interest. Sifting through such \emph{big} datasets, which can range into millions of images, and making inferences based on this data, is evolving into one of the biggest challenges in the field research community. This motivates the need for automated object recognition and image analysis tools.

This dissertation focusses on object recognition techniques capable of operating in noisy natural environments.
A series underwater object recognition  problems have been solved as means to validate the developed object recognition algorithms. 
Each technique was developed to complement the shortcomings of the existing tools available to the research community. 
At first, eigen-value based shape descriptors were tasked to solve a submerged subway car recognition problem. Despite being successful in solving this problem, the eigen-value shape descriptor method cannot leverage textural cues for object identification. This primary drawback, among other shortcomings, lead to the development of a multi-layered object recognition architecture. This multi-layered architecture was tested on an scallop enumeration problem. 60-70\% of scallop instances were successfully identified. To improve the machine learning classifier of this multi-layered framework, and also to minimize false positives, a multi-view object classification approach is proposed. This multi-view approach combines histogram-based global cues from a series of images of a target, captured from different heights, to construct a machine learning classifier. This multi-view method was successful in classifying all specimens in the available dataset. In addition to the developed object recognition methods, a low cost \gls{rov}, 
named CoopROV, was designed for underwater data collection to support research experiments.
